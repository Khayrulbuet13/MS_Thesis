\chapter{Summary and Conclusions}

\section{Conclusion}

We have coupled a computationally inexpensive supra CG model with fluctuating lattice-Boltzmann method, which enables us to simulate the entire EV squeezing process with a reasonable computational cost. Each lipid molecule is represented by one head particle and two tail particles. In-house code is used to compute the pore formation and relate the pore area with drug loading through the diffusion process. The proposed method can predict the drug loading into the EV and thus can help tune the squeezing parameters for desired drug loading. The method is free from any empirical parameter depending on operating conditions and channel geometry. Our results show that drug loading through the EV increases with increasing flow velocity, increasing EV diameter, decreasing constriction width, and increasing constriction length. An optimal value for channel width can be predicted to achieve maximum drug loading without damaging the EV. EV squeezing through nanochannel is a complicated phenomenon in biophysics; thus, the coarse-grained simulation presented in this work can provide insights into both the physics during the dynamic process and how various parameters influence drug loading. EV squeezing velocity, constriction width and length properties of the EV such as their sizes, shapes, and initial positions are parameters that can significantly change the loading resultsAnalytical equation used to calculate drug loading is one of the limitation of the proposed method that  assume pores to be sealed after 30 s. However, pore closing time can differ depending on pore radius70. We are working on developing a more efficient model that can simulate the pore closing with reasonable computational cost. The proposed model can be used to optimize squeezing parameters like flow velocity, channel geometry parameter and EV diameter to obtain preferred drug loading through EV.


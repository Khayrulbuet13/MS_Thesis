\chapter{Introduction}

The development of new drug delivery systems has been a significant area of research in recent years, with the goal of improving the efficacy and safety of therapeutic agents. One approach that has gained attention is the use of extracellular vehicles (EVs) as carriers for drug delivery. EVs are natural nanoparticles released by cells that play important roles in intercellular communication and the transfer of bioactive molecules. The use of EVs as drug carriers has several advantages, including their ability to target specific cells and tissues, their biocompatibility, and their ability to protect their cargo from degradation.

The cutting-edge technique of using microfluidic devices for drug loading into extracellular vesicles (EVs) has emerged as a game-changer in the field of drug delivery. By applying controlled levels of mechanical force to cells, EVs can be released and subsequently loaded with the desired drug cargo, resulting in a highly efficient and precise drug delivery system. This method is a remarkable feat in biomedical research as it offers a fast, reliable, and cost-effective solution for drug delivery with immense potential for clinical applications. The use of microfluidic devices has the added advantage of allowing precise control over the amount and type of drug loaded into the EVs, enabling researchers to tailor the drug delivery system to specific therapeutic needs. This groundbreaking technique offers a promising avenue for the development of novel drug delivery systems and holds the potential to revolutionize the way we treat diseases.

Drug-loaded extracellular vesicles (EVs) hold enormous promise for revolutionizing the field of drug delivery, with potential applications ranging from cancer therapy to the treatment of neurological and cardiovascular diseases. This innovative approach to drug delivery has the potential to improve patient outcomes by enabling precise and targeted delivery of therapeutics to specific sites within the body. In this study, we will delve into the underlying science behind EVs squeezing and their immense potential as drug carriers. We will also explore the optimal technique of squeezing for loading drugs into these EVs. Our discussion will shed light on the exciting possibilities of EV-based drug delivery systems, opening up new avenues for the development of advanced therapeutics with tremendous potential to transform healthcare.

\section{Motivation} \label{sec: Motivation}

Biomembranes are among the most crucial parts of every living organism. The complex structure of biomembranes is composed of various fluid-like amphiphilic phospholipid molecules, transmembrane proteins, and carbohydrates. Extracellular vesicle contains fluid-like closed structures of phospholipids suspended in solutions and have wide application in biophysics and nanomedicine due to their similarity with natural biomembranes. For instance, EVs are widely used as drug delivery vehicles\cite{a1,a2}. Being naturally derived composition and function as intracellular communication tools, they are free from several disadvantages compared to liposomal drug delivery, e.g., endosomal degradation, immune clearance, organ toxicity, and insertional mutagenesis.\cite{a3,a4,a5,a6} Commonly used drug loading methods for EVs include passive incubation and electroporation. Passive incubation has the drawback of long incubation time and low loading efficiency \cite{a7,a8,a9} where electroporation can cause significant damage to both EV and their cargos. \cite{a7,a8,a10,a11} Alternatively, nanofluidic squeezing can easily manipulate flow and samples of interest, thus showing great potential for EV drug loading\cite{a12,a13}. The process includes mixing EV and drug suspension, injection through a microfluidic device, and squeezing EVs through nanochannel to generate transient nanopores on the membrane, thus allowing targeted cargos to enter the EV. Transporting cargo through the membrane is a diffusion process that continues until the pore heals if the saturation level is not reached. 

Various experimental methods have been developed to study the diffusion of lipid bilayers\cite{a14}, deformations of vesicles\cite{a15,a16}, and membrane fusion\cite{a17}. For instance, super-resolution imaging techniques have progressed significantly to investigate the dynamics of biological membranes\cite{a18,a19}. However, the pore formation and rupture of EVs under large deformations are hard to be studied with experimental approaches due to their small scale and transient short time scales. The current experimental setup and testing conditions are still based on trial and error and are far from achieving the best loading performance. Various numerical approaches have been developed during the last three decades to complement experimental understandings of biomembranes. While All Atomic (AA) modeling techniques are still progressing in terms of developing more efficient computational algorithms, more accurate force fields and advanced sampling techniques \cite{a20,a21,a22,a23,a24}, many biological phenomena are still not in the accessible range of AA approaches. Coarse-graining AA systems can significantly reduce degrees of freedom and accelerate system dynamics to model higher lengths and longer time scales. Chemical specificity preservative CG models such as MARTINI have been extensively employed to study the diffusion of lipid molecules \cite{a25}, vesicle fusion, lipid rafting\cite{a26,a27}, transmembrane protein aggregation\cite{a28,a29}, and pore formation\cite{a30}. To further push the accessible length and time scale, various supra coarse-grained models of lipid molecules in which only a few beads represent the whole lipid molecule have been developed\cite{a31,a32,a33,a34,a35,a36}. In explicit solvent approaches, the solvent particles are modeled explicitly and the dynamics of systems are usually modeled by molecular dynamics (MD) or dissipative particle dynamics (DPD)\cite{a37}. 

\begin{figure}[htbp]
  \centering
  \includesvg[inkscapelatex=false,width=0.7\columnwidth]{Image/1.simu_process.svg}
  \caption{ Drug loading through EV}
  \vspace{0.5cm}
  \label{fig:1}
\end{figure}

In the explicit solvent methods, majority of the simulation time is devoted to the calculation of interactions among solvent particles. Some implicit solvent approaches are efficient at the cost of a reduction in accuracy due to neglecting solvent degrees of freedom. Thus, the implicit solvent model of Cooke and Deserno\cite{a33} has been widely used in the simulation of membrane bending rigidity\cite{a38}, protein aggregation\cite{a39}, pore formation\cite{a38,a40}, and membrane fission\cite{a41} due to its efficiency with high accuracy compared to other implicit solvent approaches. The implicit solvent CG models of Cooke and Deserno\cite{a33} can be coupled with fluid flow domains thermalized with stochastic fluctuations to consider the hydrodynamics effects without modeling all the solvent degrees of freedom explicitly. For instance, Atzberger et al.\cite{a42} and Wang et al.\cite{a43} developed an immersed boundary fluctuating hydrodynamics method that can be coupled with an implicit solvent CG model. Sterpone et al.\cite{a44} developed a multiscale model with a CG protein model that was coupled with a lattice-Boltzmann\cite{a45,a46} flow solver. In a similar approach, Brandner et al.\cite{a47} and Yu and Dutt\cite{a48} coupled the fluctuating lattice Boltzmann method\cite{a46} with the implicit solvent Dry MARTINI\cite{a26} approach to study the self-assembly and aggregation of lipid bilayers and deformations of lipid vesicles under high shear rate flow. In addition to particle-based approaches, continuum scale models such as boundary element methods have also been employed in modeling the deformation of lipid vesicles in fluid flow\cite{a49,a50}. Helfrich's free energy model\cite{a51} is usually used to find the bilayer response to hydrodynamic perturbations. However, such continuum scale models lack enough details to model the formation of pores or rupture and typically are only accurate at low deformation regimes. 

In this study, coupling between a minimal CG model of a lipid bilayer with the fluctuating lattice-Boltzmann method is presented. The head-tail, implicit solvent, CG model of Cooke and Deserno\cite{a33} was used to represent the lipid molecules by only one head and two tail particles. The CG particles were coupled with a thermalized lattice-Boltzmann fluid flow solver. Moreover, hydrodynamics was modeled by a computationally efficient, grid-based, lattice-Boltzmann method with good parallelization and scalability. By employing such a CG model and fluid flow solver, we could model the squeezing process of EVs with a diameter of tens of nanometers under realistic flow rates in a large computational box of a few hundred nanometers. The developed model was used to study the deformation of EVs due to squeezing, pore formation, and drug loading through the transient pores. Unlike the stain-based pore formation model\cite{a52}, the pore formation process is simulated explicitly through coarse grained molecular dynamics coupled with a Lattice-Boltzmann flow solver, yet with large dimensional and time scales far beyond the reach of traditional molecular dynamics simulations. A novel approach for predicting pore formation and drug loading through the EV  is presented by studying the effects of various squeezing parameters.  This parametric study will provide guidance for the development of better EV drug loading conditions and devices.

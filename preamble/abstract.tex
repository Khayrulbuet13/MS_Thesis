\chapter*{Abstract}
\hspace{1cm} In recent years, extracellular vesicles have emerged as a promising avenue for drug delivery. However, loading exogenous cargos into the vesicles without damaging their membrane poses a significant challenge. One commonly used method involves rapidly squeezing the vesicles through nanofluidic channels to create nanopores on the membrane, allowing for cargo loading. Unfortunately, the exact process and dynamics of nanopore formation and cargo loading through nanopores remain unknown due to the fast, transient nature of the process and the small size of the vesicles. To address this gap, we developed a comprehensive algorithm that simulates nanopore formation and predicts drug loading during extracellular vesicle squeezing. We leveraged the power of coarse-grain molecular dynamics simulations with fluid dynamics, coupling EV CG beads with implicit Fluctuating Lattice Boltzmann solvent. Our simulation analyzed the effects of various squeezing test parameters, such as EV size, flow velocity, channel width, and length, and EV properties on pore formation and drug loading efficiency. Our simulation results allowed us to generate a phase diagram to guide the design of nanochannel geometry and squeezing velocity, enabling us to generate nanopores on the membrane without damaging the EV. This approach can be used to optimize nanofluidic device configuration and flow setup to achieve optimal drug loading into EVs. Overall, this simulation method offers valuable insights into the drug delivery process and may lead to improved drug delivery platforms in the future.
